\documentclass{../assignment}

\title{\lr{HTTP}}

\begin{document}

\maketitle

\tableofcontents
\makefooter
\newpage

\raggedleft
\section{تئوری}

\subsection{سوال یک}
تفاوت های کلیدی ورژن های مختلف http شامل 0.1, 1.1, 2 و 3 را به صورت کلید واژه ای عنوان کنید.

چرا در http ورژن 3 از پروتکل UDP به عنوان پروتکل لایه ارتباط استفاده شده است؟
\subsection{سوال دو}
ارتباطات دنیای وب را می‌توان در ۲ دسته stateless و stateful قرار داد.
\linebreak
الف- این دسته بندی را به صورت مختصر توضیح دهید ( در یک پاراگراف حداکثر ۵ خطی).
\linebreak
ب- پروتکل HTTP در کدام دسته قرار می‌گیرد و برای حل این مشکلات ناشی از این دسته بندی از چه روشی استفاده می‌کند؟

\subsection{سوال سوم}
در ارتباطات HTTP، از مدل client/server استفاده می‌شود. این ارتباط یک محدودیت دارد آنهم اینکه همیشه شروع کننده ارتباط باید client باشد و جریان طبیعی اطلاعات فقط در صورتی که client درخواست اطلاعات از سرور کند از سمت سرور به کلاینت می‌شود. این ماهیت باعث می‌شود که استفاده از این پروتکل در کاربرد هایی نظیر webhook ها و چت و alerting چالشی باشد. 
\linebreak
الف- این چالش را توضیح دهید. (خلاصه)
\linebreak
ب- درمورد راهکار(ها)یی که برای مقابله با این چالش ارائه شده اند تحقیق کنید و توضیح دهید. (هر راهکار احتمالی را در یک پاراگراف توضیح دهید)
\subsection{سوال چهارم}
در URL زیر اجزا را مشخص کنید.

\hfill
ss://asghar:1234!!@ss.myproxy.com:1234\#shadowSocks1

\subsection{سوال پنجم}
برای هر یک از سناریو های زیر در یک \lr{Rest\space\space API} کدام \lr{http\space\space status\space\space code} مناسب است. (دقت کنید که ممکن است برای یک سناریو چند کد مختلف مناسب باشند که در این حالت بنابر تشخیص خود دلیل انتخاب را بیان کنید)
\linebreak
از این لینک به عنوان مرجع برای کد ها استفاده کنید:

\hfill
https://developer.mozilla.org/en-US/docs/Web/HTTP/Status
\linebreak
\linebreak
1- دیتابیس وبسایت شما مشکل پیدا کرده و درخواست های لاگین انجام نمی‌شوند.
\linebreak
2- نام‌کاربری ای که کاربر وارد کرده داخل سایت وجود ندارد و لاگین موفقیت آمیز نبود.
\linebreak
3- وبسایت به دامنه دیگری منتقل شده است.
\linebreak
4- تعداد درخواست هایی که از طرف این کاربر ارسال شده از نرخ مجاز بیشتر است.
\linebreak
5- درخواست تمدید توکن (JWT) موفقیت آمیز بوده و توکن تمدید شده داخل هدر بازگردانده شده.
\linebreak
6- کاربر از رنج آیپی ممنوعه برای دسترسی به سرویس است. (تحریم شده)
\subsection{سوال ششم}
برای هر یک از سناریو های گفته شده نوع پروکسی استفاده شده را مشخص کنید.
\linebreak
\linebreak
1- استفاده از سرور های داخلی دانشگاه بجای سرور های مرکزی برای دریافت پکیج ها از پکیج منیجر (apt)
\linebreak
2- تقسیم درخواست ها بر اساس مبدا جغرافیایی درخواست ها
\linebreak
3-اضافه کردن قابلیت احراز هویت برای یک سرویس بسیار قدیمی 
\section{عملی}

۱- پیاده سازی یک \lr{authentication \space\space  proxy} به این صورت که همه درخواست های یک سرویس رو از این پروکسی رد بکنیم. سرویس ما باید با استفاده از JWT درخواست ها را احراز هویت بکنه و در صورت موفقیت آمیز بودن احراز هویت درخواست ها را به سرویس فرستاده و پاسخ ها را دریافت و به کاربر اعلام کند.

این سرویس میتونه توسط خود دانشجو برنامه نویسی بشه و یا توسط پیکربندی وب سرور ها و \lr{reverse\space\space proxy} های آماده انجام بشه…




\end{document}
