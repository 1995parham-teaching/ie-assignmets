\documentclass{../assignment}

\title{\lr{HTTP}}

\begin{document}

\maketitle

\tableofcontents
\newpage

\قسمت{تئوری}

\زیرقسمت{سوال یک}

تفاوت‌های کلیدی نسخه‌های مختلف \متن‌لاتین{http} شامل 0.1, 1.1, 2 و 3 را به صورت کلید واژه‌ای عنوان کنید.

چرا در \متن‌لاتین{http} نسخه 3 از پروتکل UDP به عنوان پروتکل لایه ارتباط استفاده شده است؟

\زیرقسمت{سوال دو}

ارتباطات دنیای وب را می‌توان در ۲ دسته \متن‌لاتین{stateless} و \متن‌لاتین{stateful} قرار داد.

\شروع{شمارش}[ا -]
\فقره این دسته بندی را به صورت مختصر توضیح دهید (در یک پاراگراف حداکثر ۵ خطی).
\فقره پروتکل HTTP در کدام دسته قرار می‌گیرد و برای حل این مشکلات ناشی از این دسته بندی از چه روشی استفاده می‌کند؟
\پایان{شمارش}

\زیرقسمت{سوال سوم}

در ارتباطات HTTP، از مدل client/server استفاده می‌شود. این ارتباط یک محدودیت دارد آنهم اینکه همیشه شروع کننده ارتباط باید client باشد و جریان طبیعی اطلاعات فقط در صورتی که client درخواست اطلاعات از سرور کند از سمت سرور به کلاینت می‌شود. این ماهیت باعث می‌شود که استفاده از این پروتکل در کاربرد هایی نظیر webhook ها و چت و alerting چالشی باشد.

\شروع{شمارش}[ا -]
\فقره این چالش را توضیح دهید. (خلاصه)
\فقره درمورد راهکار(ها)یی که برای مقابله با این چالش ارائه شده اند تحقیق کنید و توضیح دهید. (هر راهکار احتمالی را در یک پاراگراف توضیح دهید)
\پایان{شمارش}

\زیرقسمت{سوال چهارم}

در \متن‌لاتین{URL} زیر اجزا را مشخص کنید.

\begin{latin}
\begin{verbatim}
ss://asghar:1234!!@ss.myproxy.com:1234\#shadowSocks1
\end{verbatim}
\end{latin}

\subsection{سوال پنجم}
برای هر یک از سناریو های زیر در یک \متن‌لاتین{ReST API} کدام \متن‌لاتین{httpe statuse code} مناسب است.
(دقت کنید که ممکن است برای یک سناریو چند کد مختلف مناسب باشند که در این حالت بنابر تشخیص خود، دلیل انتخاب را بیان کنید)
از \تارنما{https://developer.mozilla.org/en-US/docs/Web/HTTP/Status}{اینجا} به عنوان مرجع برای کد ها استفاده کنید.

\شروع{شمارش}
\فقره دیتابیس وبسایت شما مشکل پیدا کرده و درخواست های لاگین انجام نمی‌شوند.
\فقره نام‌کاربری ای که کاربر وارد کرده داخل سایت وجود ندارد و لاگین موفقیت آمیز نبود.
\فقره وبسایت به دامنه دیگری منتقل شده است.
\فقره تعداد درخواست هایی که از طرف این کاربر ارسال شده از نرخ مجاز بیشتر است.
\فقره درخواست تمدید توکن \متن‌لاتین{(JWT)} موفقیت آمیز بوده و توکن تمدید شده داخل هدر بازگردانده شده.
\فقره کاربر از رنج \متن‌لاتین{IP} ممنوعه برای دسترسی به سرویس است. (تحریم شده)
\پایان{شمارش}

\زیرقسمت{سوال ششم}

برای هر یک از سناریو های گفته شده نوع پروکسی استفاده شده را مشخص کنید.

\شروع{شمارش}
\فقره استفاده از سرور های داخلی دانشگاه بجای سرور های مرکزی برای دریافت پکیج ها از پکیج منیجر (مانند \متن‌لاتین{apt} برای \متن‌لاتین{Ubuntu})
\فقره تقسیم درخواست ها بر اساس مبدا جغرافیایی درخواست ها
\فقره اضافه کردن قابلیت احراز هویت برای یک سرویس بسیار قدیمی
\پایان{شمارش}

\قسمت{عملی}

\زیرقسمت{احراز هویت با پراکسی}
پیاده سازی یک \lr{authentication \space\space  proxy} به این صورت که همه درخواست های یک سرویس رو از این پروکسی رد بکنیم. سرویس ما باید با استفاده از JWT درخواست ها را احراز هویت بکنه و در صورت موفقیت آمیز بودن احراز هویت درخواست ها را به سرویس فرستاده و پاسخ ها را دریافت و به کاربر اعلام کند.

این سرویس میتونه توسط خود دانشجو برنامه نویسی بشه و یا توسط پیکربندی وب سرور ها و \lr{reverse\space\space proxy} های آماده انجام بشه…

\makefooter

\end{document}
