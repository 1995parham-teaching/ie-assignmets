\documentclass{../assignment}

\عنوان{\متن‌لاتین{Backend}}

\begin{document}

\عنوان‌ساز

\فهرست‌مطالب

\قسمت{مقدمه}

در این تمرین قصد داریم که یک سرویس ارسال ایمیل را توسعه دهیم. در این پروژه سعی داریم مفاهیم زیر را پوشش دهیم:
\شروع{فقرات}
\فقره \متن‌لاتین{endpoint}های \متن‌لاتین{CRUD} برای یک \متن‌لاتین{resource} از پیش تعریف شده
\فقره ارتباط با یک پایگاه داده جهت ذخیره سازی اطلاعات
\فقره ارتباط با یک \متن‌لاتین{ReST API} ثانوی جهت ارسال ایمیل
\پایان{فقرات}

در ادامه، مراحل و گام‌های لازم برای رسیدن به هر یک از این اهداف را شرح می‌دهیم.
استفاده از هر زبان برنامه‌نویسی برای انجام این تمرین \متن‌سیاه{آزاد} است، ولی مراحل و گام‌ها برای زبان \متن‌لاتین{Go} تعریف شده‌اند.

\قسمت{سناریو}

فرض کنید می‌خواهیم برای دانشجویان دانشگاه، ایمیل‌های از پیش تعریف شده‌ای برای هر درس ارسال کنیم.
برای اینکار نیاز به یک میکروسرویس داریم، که توسعه آن به شما واگذار شده است.

\قسمت{طراحی و پیاده‌سازی مدل‌ها}

در این قسمت می‌بایست مدل‌های \متن‌سیاه{دانشجو} و \متن‌سیاه{کلاس} را تعریف و پیاده سازی کنیم.
جهت سادگی و جلوگیری از پیچیدگی‌های طراحی پایگاه داده فرض کنید که \متن‌سیاه{هر دانشجو فقط می‌تواند عضو یک کلاس باشد}.
اطلاعاتی که برای هر یک از این مدل‌ها باید ذخیره کنیم:

\شروع{فقرات}

\فقره داتشجو

\شروع{فقرات}
\فقره نام و نام خانوادگی
\فقره ایمیل
\فقره کلاس
\فقره نمره
\پایان{فقرات}

\فقره کلاس

\شروع{فقرات}
\فقره نام درس
\فقره استاد
\پایان{فقرات}

\پایان{فقرات}

مواردی که باید انجام دهید:

\شروع{شمارش}

\فقره برای هر یک از مدل‌ها داخل زبان برنامه نویسی مورد نظر یک کلاس/فایل تعریف کنید.
\فقره یک مکانیزم ذخیره‌سازی دائم (بر روی دیسک) برای مدل‌های خود پیاده‌سازی کنید.
\شروع{شمارش}
\فقره برای انجام این کار می‌توانید از \متن‌لاتین{ORM}ها استفاده کنید. \متن‌لاتین{ORM} پیشنهادی برای زبان \متن‌لاتین{Go}، کتابخانه \تارنما{https://github.com/go-gorm/gorm}{\متن‌لاتین{gorm}} است.
\فقره استفاده از مکانیزم‌های ذخیره‌سازی مستقیم \متن‌لاتین{object}ها بر روی دیسک قابل قبول نیست ولی می‌توانید از هر کتابخانه و روشی که اطلاعات را \متن‌لاتین{serialize} شده ذخیره می‌کند، استفاده نمایید.
\پایان{شمارش}
\فقره توابع لازم جهت دریافت، ذخیره‌سازی، بروزرسانی و حذف برای این مدل ها بنویسید.
\شروع{شمارش}
\فقره برای حذف نیاز به \متن‌لاتین{cascade delete} نیست و چنین حالتی در ارزیابی برنامه نیست. یعنی فرضا اگر یک کلاس حذف شد نیازی به حذف رکوردهای مربوط دانشجوهای مربوط به آن درس نیست.
\پایان{شمارش}

\پایان{شمارش}

\قسمت{پیاده‌سازی رابط برنامه}

در این بخش می‌خواهیم که یک رابط REST به کاربران خود ارائه دهیم. کاربر برنامه از این طریق با برنامه ما ارتباط برقرار می‌کند و درخواست‌های خود را ارسال و پاسخ می‌گیرد. امکاناتی که در این بخش باید به کاربر خود ارائه دهیم شامل دخل و تصرف داخل مدل‌ها و امکان ارسال ایمیل است. دقت داشته باشید که استفاده از این ویژگی‌ها باید توسط یک مکانیزم احراز هویت کنترل شوند.

مواردی که باید انجام دهید:

\شروع{شمارش}

\فقره پیاده‌سازی یک مکانیزم احراز هویت برای کاربر استفاده کننده از \متن‌لاتین{API}

\شروع{فقرات}

\فقره مکانیزم‌های پیشنهادی‌: \متن‌لاتین{jwt}، \متن‌لاتین{basic auth}، \متن‌لاتین{api-key}

\پایان{فقرات}

\فقره پیاده‌سازی \متن‌لاتین{endpoint}های مربوط به مدل‌ها (دانشجو و کلاس)

\شروع{فقرات}

\فقره رد و بدل اطلاعات باید با استاندارد \متن‌لاتین{JSON} انجام شود.

\فقره برای مدل دانشجو هنگام پردازش درخواست محدودیت‌های زیر را در نظر بگیرید:

\شروع{فقرات}

\فقره فرمت ایمیل با \متن‌لاتین{regex} چک شود.
\فقره نمره وارد شده یک عدد طبیعی بین ۰ تا ۲۰ باشد.

\پایان{فقرات}

\فقره درخواست ساخت دانشجو می‌توانید ورودی را لیستی از دانشجویان در نظر بگیرد.

\پایان{فقرات}

\فقره پیاده‌سازی \متن‌لاتین{endpoint} ارسال ایمیل

\شروع{فقرات}

\فقره برای ارسال ایمیل کاربر کافیست ID کلاس را ارسال کند. برنامه بایستی نمره دانشجو، نام دانشجو، نام کلاس و استاد درس را در ایمیل ارسالی قرار دهد.

\فقره پاسخ endpoint باید شامل وضعیت تمام ایمیل‌های موجود در کلاس باشد.

\پایان{فقرات}

\فقره در پاسخ دادن به درخواست‌ها \متن‌لاتین{HTTP Status Code} مناسب استفاده نمایید.

\فقره \رنگ‌متن{قرمز}{امتیازی.} یک اندپوینت برای اضافه کردن کلاس درس به وسیله آپلود فایل اضافه کنید.

\شروع{فقرات}

\فقره فرمت‌های قابل قبول شامل \متن‌لاتین{csv}، \متن‌لاتین{excel} و \متن‌لاتین{xml} است.

\پایان{فقرات}

\پایان{شمارش}

برای فرمت عملیات‌ها، درخواست ها و پاسخ ها یک فایل \متن‌لاتین{Postman Collection} به پیوست تمرین ارسال شده است، توصیه می‌شود که از این فایل جهت تعریف \متن‌لاتین{API} خود استفاده کنید.

پیشنهاد ما برای پیاده‌سازی این \متن‌لاتین{API} استفاده از چهارچوب \تارنما{https://github.com/labstack/echo/}{\متن‌لاتین{echo}} یا \تارنما{https://github.com/gofiber/fiber/}{\متن‌لاتین{gofiber}} است.
برای تعریف هر قابلیت \متن‌لاتین{API} از یک \متن‌لاتین{handler function} استفاده می‌کنیم.
یک نمونه از این توابع برای ساخت دانشجو (به صورت لیستی) در قطعه کد \رجوع{قطعه‌کد: نمونه‌ای از ساخت دانشجویان به صورت لیستی به همراه صحت‌سنجی} آورده شده است که در آن از کتابخانه \تارنما{https://github.com/go-playground/validator}{\متن‌لاتین{go validator}} جهت اعمال محدودیت‌های دانشجو استفاده می‌کنیم.

\begin{listing}

\شرح{نمونه‌ای از ساخت دانشجویان به صورت لیستی به همراه صحت‌سنجی}
\برچسب{قطعه‌کد: نمونه‌ای از ساخت دانشجویان به صورت لیستی به همراه صحت‌سنجی}

\begin{latin}
\begin{minted}[bgcolor=Black]{go}
// considering having StudentRepo defined in model package
// with SaveForClass method.
type StudentHandler struct {
  StudentRepo model.StudentRepo
  Validator   *validator.Validate
}

type Student struct {
  Name    string  `json:"name" validate:"required"`
  Email   string  `json:"email" validate:"required,email"`
  ClassID int64   `json:"class_id" validate:"required"`
  Score   float64 `json:"score" validate:"required,gte=0,lte=20"`
}

type StudentCreate struct {
  ClassID  int64     `json:"class_id" validate:"required"`
  Students []Student `json:"students" validate:"required,dive"`
}

func (s StudentHandler) Create(c echo.Context) error {
  req := new(StudentCreate)

  if err := c.Bind(req); err != nil {
     return echo.ErrBadRequest
  }

  if err := s.Validator.Struct(req); err != nil {
     return echo.ErrBadRequest
  }

  err := s.StudentRepo.SaveForClass(req.ClassID, req.Students)
  if err != nil {
     return echo.ErrInternalServerError
  }

  return c.NoContent(http.StatusCreated)
}
\end{minted}
\end{latin}

\end{listing}

\قسمت{پیاده سازی کلاینت ایمیل}

ما برای ارسال ایمیل از یک ارائه دهنده خدمات ایمیل استفاده خواهیم کرد. شما در انتخاب و پیاده سازی این بخش محدودیتی ندارید. ولی پیشنهاد ما برای این بخش استفاده از \متن‌لاتین{gmail} شخصی برای این‌کار است.

برای این کار ابتدا اگر حساب شما دارای \متن‌لاتین{2FA} است، مراحل زیر را برای دریافت \متن‌لاتین{application password} دنبال کنید:

\شروع{فقرات}
\فقره از طریق این لینک به تنظیمات حساب خود وارد شوید.
\فقره از منو سمت چپ وارد قسمت \متن‌لاتین{Security} شوید.
\فقره در صفحه باز شده به دنبال گزینه \متن‌لاتین{App Passwords} بگردید و وارد این بخش شوید.
\فقره در صفحه باز شده مانند شکل \رجوع{} گزینه \متن‌لاتین{Mail} و \متن‌لاتین{Other Device} را انتخاب کنید

\فقره یک نام برای پسورد خود انتخاب کنید.
پسورد تولید شده خود را کپی کنید و داخل برنامه بجای کلمه عبور ایمیل خود استفاده کنید.
\پایان{فقرات}

ما برای ارسال ایمیل از پروتکل \متن‌لاتین{smtp} استفاده خواهیم کرد، همچنین هیچ محدودیتی در استفاده از کتابخانه های مختلف زبان برنامه نویسی نداریم.
نمونه یک برنامه کامل جهت ارسال ایمیل در قطعه کد \رجوع{قطعه‌کد: نمونه‌ای از ارسال ایمیل به وسیله‌ی پروتکل smtp} آورده شده است.

\begin{listing}

\شرح{نمونه‌ای از ارسال ایمیل به وسیله‌ی پروتکل \متن‌لاتین{smtp}}
\برچسب{قطعه‌کد: نمونه‌ای از ارسال ایمیل به وسیله‌ی پروتکل smtp}

\begin{latin}
\begin{minted}[bgcolor=Black]{go}
package main

import (
   "fmt"
   "log"
   "strings"

   "github.com/emersion/go-sasl"
   "github.com/emersion/go-smtp"
)

const (
   MailUser        = "xxxxx@gmail.com"
   MailPass        = "xxxxxxxxxxx"
   GmailSMTPServer = "smtp.gmail.com:587"
)

func composeEmail(sender, recepient, subject, body string) []byte {
  return []byte(fmt.Sprintf(
    "From: %s\r\nTo: %s\r\nSubject: %s\r\n\r\n%s\r\n",
    sender, recepient, subject, body,
  ))
}

func main() {
   // Setup authentication information.
   auth := sasl.NewPlainClient("", MailUser, MailPass)

   to := []string{"recepient@gmail.com"}
   msg := strings.NewReader(
    string(composeEmail(MailUser, to[0], "Test Subject", "Test Body"),
   ))

   err := smtp.SendMail(GmailSMTPServer, auth, MailUser, to, msg)
   if err != nil {
       log.Fatal(err)
   }
}
\end{minted}
\end{latin}

\end{listing}

مواردی که برای این قسمت باید انجام دهید:

\شروع{شمارش}
\فقره پیاده‌سازی یک کلاینت جهت ارسال ایمیل به لیستی از دریافت کننده‌ها

\شروع{شمارش}
\فقره کلاینت شما باید وضعیت ارسال هر ایمیل را به صورت جداگانه گزارش کند.
یعنی برای یک لیست شامل ۳ ایمیل، وضعیت هر ۳ ایمیل در قالب یک آرایه از گزارش‌ها برگردانده شود.
\پایان{شمارش}

\فقره متن ارسالی در \متن‌لاتین{Body} ایمیل باید شامل موارد خواسته شده در قسمت \متن‌لاتین{API} باشد.

\فقره \رنگ‌متن{قرمز}{امتیازی.} ارسال ایمیل ها را به صورت موازی انجام دهید که ارسال لیست طولانی‌ای از ایمیل‌ها مدت زمان زیادی نبرد.

\پایان{شمارش}

\قسمت{مقررات اجرا و تحویل}

\شروع{فقرات}

\فقره برنامه شما باید در نهایت قابلیت ساخت یک کلاس آزمایشی برای چند دانشجو را داشته باشد.
\فقره اکیدا از قراردادن \متن‌لاتین{password} خود داخل فایل ارسالی خودداری نمایید.
\فقره این تمرین همانند تمرین عملی اول تحویل آنلاین دارد.
\فقره استفاده از \متن‌لاتین{Docker} جهت اجرای برنامه نمره امتیازی دارد.

\پایان{فقرات}

\پایان‌ساز

\end{document}
