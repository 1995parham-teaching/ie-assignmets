\documentclass{../assignment}

\عنوان{\متن‌لاتین{Backend}}

\begin{document}

\عنوان‌ساز

\فهرست‌مطالب

\قسمت{مقدمه}

در این تمرین قصد داریم که یک سرویس با زبان برنامه‌نویسی \متن‌لاتین{Go} یا \متن‌سیاه{هر زبان دیگری} برای ارائه نظارت بر \متن‌لاتین{endpoint}های \متن‌لاتین{http} در بازه‌های زمانی \متن‌سیاه{قابل تنظیم} (برای مثال هر ۳۰ثانیه، ۱ دقیقه، ۵ دقیقه) توسعه دهیم. این سرویس یک درخواست \متن‌لاتین{http} برای \متن‌لاتین{endpoint} می‌فرستد
و گزارش کد وضعیت پاسخ را ذخیره می‌کند. هر آدرس اینترنتی باید یک آستانه خطا که حداکثر تعداد خطای قابل تحمل را نشان می دهد، داشته باشد که پس از گذر از این تعداد سرویس ما باید هشداری را برای کاربری که آدرس اینترنتی به او تعلق دارد، تولید کند.

فراخوان موفقیت آمیز \متن‌لاتین{http} با کد وضعیت \متن‌لاتین{2xx} مشخص می‌شود و فراخوانی که با موفقیت همراه نبوده با وضعیت کدی غیر از \متن‌لاتین{2xx} مشخص خواهد شد.

\قسمت{پایگاه داده}

برای استفاده از هر نوع پایگاه داده دست شما باز خواهد بود.

\قسمت{طراحی و پیاده‌سازی}

در این تمرین باید یک رابط برنامه‌نویسی کاربردی \متن‌لاتین{RESTFul} با ویژگی‌های زیر طراحی و تعریف کنید.

\زیرقسمت{احراز هویت}

در این قسمت از \متن‌لاتین{JWT Token} برای احراز هویت کاربران استفاده می‌شود. در رابطه با \متن‌لاتین{JWT} در بحث \متن‌لاتین{HTTP} صحبت شده است اما شما نیاز دارید
این موارد را خودتان پیاده‌سازی کنید، یعنی توکن را در سرآیند\پانویس{Header} \متن‌لاتین{Authorization} دریافت کرده و آن را تایید کنید و در هنگام ورود هم یک \متن‌لاتین{Token} تولید کنید.
با توجه به پیچیدگی‌های تولید و تایید \متن‌لاتین{Token}ها بهتر است در زبان \متن‌لاتین{Go} از کتابخانه‌ی \تارنما{https://github.com/golang-jwt/jwt}{\متن‌لاتین{golang-jwt}} استفاده کنید.

\زیرقسمت{کاربران}

در این تمرین برای مدیریت و احراز هویت کاربران رویه‌ی ساده‌ای را در نظر می‌گبریم که کاربران در ابتدا بتوانند بعد از ثبت نام وارد شده و \متن‌لاتین {Token} دریافت نمایند
تا بتوانند با استفاده از آن آدرس اینترنتی بسازند، آن را آپدیت کنند،‌ وضعیت فعلی آدرس اینترنتی و هشدار‌های که برای آن‌ها موجود هست رو دریافت کنند.

\زیرقسمت{\متن‌لاتین{endpoint}های کاربران}

\شروع{فقرات}
\فقره ساخت کاربر جدید (عمومی)
\فقره تولید \متن‌لاتین{Token} برای کاربران (عمومی)
\پایان{فقرات}

\زیرقسمت{آدرس اینترنتی}

هر آدرس اینترنتی بایستی یک آستانه خطا که حداکثر تعداد خطا را نشان می دهد، داشته باشد که پس از گذر از این تعداد، سرویس باید هشداری را برای کاربری که آدرس اینترنتی به آن تعلق دارد، تولید کند.
هر کاربر نهایتا می‌تواند تا حداکثر ۲۰ آدرس اینترنتی بسازد.
در نظر داشته باشید که در رابطه با هشدارها در ادامه تعریف پروژه بیشتر صحبت خواهد شد.

\زیرقسمت{\متن‌لاتین{endpoint}های آدرس‌های اینترنتی}

\شروع{فقرات}
\فقره ساخت آدرس اینترنتی جدید (نیازمند احراز هویت)
\فقره دریافت تمامی آدرس‌های اینترنتی کاربر (نیازمند احراز هویت)
\فقره آمار روزانه هر آدرس اینترنتی مشخص (موفق یا ناموفق). برای مثال تعداد فراخوانی ‌های موفق یا غیرموفق در طول روز (نیازمند احراز هویت)
\پایان{فقرات}

\زیرقسمت{هشدارها}

زمانی که تعداد خطاها از آستانه حداکثری تعداد خطاهای تعریف شده برای آدرس‌ اینترنتی گذر کرد، سرویس مورد نظر بایستی یک هشدار برای آن آدرس اینترنتی تعریف کند.
کاربر نیز بایستی قادر به دیدن این هشدارها باشد.

\زیرقسمت{\متن‌لاتین{endpoint}های هشدار}

\شروع{فقرات}
\فقره دریافت هشدارهای مخصوص به آدرس اینترنتی (نیازمند احراز هویت)
\پایان{فقرات}

\متن‌سیاه{توجه: تمامی سرویس‌ها که به صورت عمومی هست بایستی قابل دسترس همه باشد و سرویس‌های نیازمند احراز هویت باید برای کسانی که \متن‌لاتین{token} دارند قابل دسترس باشند.}


\قسمت{مقررات تحویل}

یک مخزن گیت‌هاب جدید برای این تمرین بسازید و آن را با تیم تدریسیاری به اشتراک بگذارید.

\شروع{فقرات}
\فقره تلاش کنید، رابط کاربردی برنامه‌نویسی با استانداردهای \متن‌لاتین{RESTFul} طراحی کنید.
\فقره ازکامت‌های مختصر و کامل استفاده کنید.
\فقره این تمرین تحویل آنلاین دارد.
\فقره استفاده از \متن‌لاتین{Docker} جهت اجرای برنامه نمره امتیازی دارد.
\فقره مستندسازی در مورد سرویس‌های خود در \متن‌لاتین{README.md} جهت مشخص نمودن سرویس‌های خود بنویسید.
\پایان{فقرات}

\پایان‌ساز

\end{document}
