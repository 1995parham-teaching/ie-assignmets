\documentclass{../assignment}

\عنوان{\متن‌لاتین{Backend}}

\شروع{نوشتار}

\عنوان‌ساز

\فهرست‌مطالب

\قسمت{مقدمه}

در این تمرین قصد داریم که یک سرویس با زبان برنامه‌نویسی \متن‌لاتین‌ {go} یا هر زبان دیگه برای ارائه نظارت بر \متن‌لاتین‌ {endpoint} \متن‌لاتین {http} ها در بازه زمانی‌های قابل تغییر (برای مثال هر ۳۰ثانیه، ۱ دقیقه، ۵ دقیقه) این سرویس یک درخواست \متن‌لاتین {http} برای \متن‌لاتین {endpoint} فرستاده می‌شود و گزارش کد وضعیت پاسخ و .... دهیم. هر آدرس اینترنتی بایستی یک آستانه خطا که حداکثر تعداد خطا را نشان می دهد داشته باشد که پس از گذر از این تعداد آن سرویس باید هشداری را برای کاربری که آدرس اینترنتی به آن تعلق دارد راه اندازی کند.

فراخوان موفقیت آمیز \متن‌لاتین {http} با کد وضعیت ××۲ مشخص می‌شود و فراخوانی که با موفقیت همراه نبوده با وضعیت کدی غیر از ××۲ مشخص خواهد شد.


\قسمت{پایگاه داده}
برای استفاده از هر نوع پایگاه داده دست شما باز خواهد بود.

\قسمت{طراحی و پیاده‌سازی }
در این تمرین بایستی یک رابط برنامه‌نویسی کاربردی \متن‌لاتین {RESTFul} با ویژگی زیر طراحی و تعریف کنید.

\شروع{فقرات}

\فقره احراز هویت

در این قسمت از \متن‌لاتین {JWT Token} برای احراز هویت کاربران مورد استفاده قرار می‌گیرد.

\فقره کاربران

در این تمرین برای مدیریت و احراز هویتی ساده برای کاربران در نظر می‌گبریم که کاربران در ابتدا بتوانند بعد از ثبت نام و دریافت \متن‌لاتین {token} تا بتوانند با استفاده از آن آدرس اینترنتی بسازند، آن را آپدیت کنند،‌ وضعیت فعلی آدرس اینترنتی و هشدار‌های که برای آن‌ها موجود هست رو دریافت کنند.

\newpage

\فقره \متن‌لاتین {endpoint}
\شروع{فقرات}
\فقره ساخت کاربر جدید (عمومی)
\فقره تولید \متن‌لاتین {Token} برای کاربران (عمومی)
\پایان{فقرات}

\فقره آدرس اینترنتی

هر آدرس اینترنتی بایستی یک آستانه خطا که حداکثر تعداد خطا را نشان می دهد داشته باشد که پس از گذر از این تعداد آن سرویس باید هشداری را برای کاربری که آدرس اینترنتی به آن تعلق دارد راه اندازی کند.

هر کاربر نهایتا می‌تواند تا حداکثر ۲۰ آدرس اینترنتی بسازد.

\فقره \متن‌لاتین {endpoint}
\شروع{فقرات}
\فقره ساخت آدرس اینترنتی جدید (نیازمند احراز هویت)
\فقره دریافت تمامی آدرس‌های اینترنتی کاربر (نیازمند احراز هویت)
\فقره آمار روزانه هر آدرس اینترنتی مشخص (موفق یا ناموفق). برای مثال تعداد فراخوانی ‌های موفق یا غیرموفق در طول روز (نیازمند احراز هویت)
\پایان{فقرات}

\فقره هشدارها
زمانی که تعداد خطاها از آستانه حداکثری تعداد خطاهای تعریف شده برای آدرس‌ اینترنتی گذر کرد سرویس مورد نظر بایستی یک هشدار برای آن آدرس اینترنتی تعریف شود. کاربر نیز بایستی قادر به دیدن این هشدارها باشد.

\فقره \متن‌لاتین {endpoint}
\شروع{فقرات}
\فقره دریافت هشدارهای مخصوص به آدرس اینترنتی (نیازمند احراز هویت)
\پایان{فقرات}

\پایان{فقرات}
\متن‌سیاه {توجه: تمامی سرویس‌ها که به صورت عمومی هست بایستی قابل دسترس همه باشد و سرویس‌های نیازمند احراز هویت باید برای کسانی که \متن‌لاتین {token} دارند قابل دسترس باشند.}
\متن‌سیاه


\قسمت{مقررات اجرا و تحویل}
یک مخزن گیت‌هاب جدید برای این تمرین بسازید و آن را با ما به اشتراک بگذارید.
\شروع{فقرات}
\فقره رابط کاربردی برنامه‌نویسی با استانداردهای \متن‌لاتین {RESTFul} طراحی کنید.
\فقره ازکامت‌های مختصر و کامل استفاده کنید.
\فقره این تمرین همانند تمرین عملی اول تحویل آنلاین دارد.
\فقره استفاده از \متن‌لاتین{Docker} جهت اجرای برنامه نمره امتیازی دارد.
\فقره مستندسازی در مورد سرویس‌های خود در \متن‌لاتین {README.md} جهت مشخص نمودن سرویس‌های خود بنویسید.
\پایان{فقرات}

\پایان‌ساز

\پایان{نوشتار}
